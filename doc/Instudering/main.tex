\documentclass{article}
\usepackage{graphicx} % Required for inserting images

\title{InduderingsfrågorIE1332}
\author{viktor.kung }
\date{April 2023}

\begin{document}

\maketitle

\section{Kapitel 1}

\begin{itemize}
    \item 1.1 Förklara vad som menas med EMC

	EMC är ett seriöst och uppåt trendande problem som handlar om
	elektromagnetisk störning och kan ses som ett form av miljöförorening.
	Detta kan t.ex vara att ett elektrisk krets stör andra elektriska
	kretsar eller tvärt om tar emot störningar från andra enheter. Det kan
	vara allt små störningar på t.ex en sprakande radiosändning till
	bedrövliga konsekvenser som fel inom säkerhetskontroll system.

\item 1.2 Förklara vad som menas med EMC‐gapet.

	EMC-gapet menar på den ökade sårbarheten av elektronisk utrustning för
	elektromagnetisk störning. Detta medför också en ökad elektromagnetisk
	miljöstörning. En del detta sker är pga det ökande användadet av VLSI
	(very large scale integration) inom mikroprocessorer. Det som är
	fördelen med denna teknik är att den använder mindre energi men är
	istället mer sårbar till störningar. En annan faktor är ökandet av
	radiokomunikation, vilket har en högre mottaglighet för
	elektronikproduker att störas. Spridningen av digital elektronik har
	lett till en ökning av låg-nivå utsläpp vilket kan påverka
	radioupptagningsförmågan. Vilket kallas för "smog"

\end{itemize}

\section{Kapitel 2, 3, 4, 5 och 6}

\begin{itemize}
    \item 2.1 Vilka delar innehåller EMC‐direktivet?

        Inom EMC-direktivet så finns flera punkter bland annat dessa som publicerades i en s.k. "Blue Guide".  Dessa punker är följande. 
        \begin{itemize}
            \item "the scope" Direktivets tillämpningsområde
            \item Ett \textit{uttalande} av de mest väsentliga kraven.
            \item \textit{Metoderna} för att tillfredställa de väsentliga kraven.
            \item Hur \textit{bevisen på överensstämmelse} kommer att tillhandahållas.
            \item Vilka \textit{Övergångsbestämmelser} som tillåts.
            \item Ett påstående för att säkerhetställa till \textit{fri cirkulation}. 
            \item Ett \textit{förfarande i fråga om skyddsåtgärder} för att tillåta medlemsstater rätten att dra tillbaka en produkt som inte uppfyller de väsentliga kraven.
        \end{itemize}
        
        Det har skett vissa förändringar med tiden gällande direktiven men det själva huvudsakliga konsensus har varit det samma. Att den elektromagnetiska störningen som den genererar inte ska överstiga en gräns där radio och tele-kommunikation inte kan fungera som avsett. Samt att det ska finnas en nivå av immunitet mot elektromagnetisk störning som gör det möjligt att enheten ska kunde fungera utan oacceptabel försämring.   
            
    \item 2.2 Vad krävs för att få sätta CE‐märke på sin produkt?

        Det finns flera olika direktiv som beroende på vilken typ av produkt CE-märkningen ska sättas på. Några av kraven som finns för att få sätta  ett CE-märke på sin produkt är:
        \begin{itemize}
            \item att det ska finnas skydd mot elchock.
            \item att det ska finnas skydd mot elbrand.
            \item att produkten inte ska störa eller slå ut andra elektriska produkter, så kallad elektromagnetisk kompatibilitet (EMC).
        \end{itemize}
    
    \item 2.3 Vad menas med Notified Body?

    "Notified Body" eller på svenska anmälda organ är organ som används för att göra en bedömning av överensstämmelse som anmälts av medlemsstaterna. Dessa används som en opartisk och kunnig part för att utföra relevanta bedömningar av produkter som ska släppas på EUs marknad. 

    \item SE.1 Vad menas med produktägare? Vilket ansvar har produktägaren?

        Enligt Elektronikhandboken definieras produktägaren som det företag som äger rättigheter till och har produktansvaret för den produkten där elektroniken som ska produceras ingår. Dess ansvar är att se till att trillräcklig och rätt information finns tillgänglig för alla som behöver den senare i kedjan så att produkten motsvarar de uppställda förväntningarna med rätt funktionalitet. Den ansvarar också för produkten uppfyller alla tillämpliga direktiv och standarder och att den har de certiferingar som krävs för den marknad som produkten ska säljas på.
    
    \item SE.2 Vad menas med DfX – Design for Excellence?

        Dfx handlar om att fokusera på kvalitet, tillförlitlighet, producerbarhet, livslängd samt uppfyllande av direktiv och regelverk under konstruktionsfasen av elektronikprodukten. 
    
    \item SE.3 Förklara vad som regleras av följande direktiv och förordningar: RoHS, WEEE, Ecodesign, REACH,
    EMC, LVD, RED

        \begin{itemize}
            \item RoHS

                RoHS (restriction of hazardous substances) är ett direktiv från EU vars ändamål är att minska användadet av farliga ämnen i elektronikprodukter då elektronikavfallen ökar. Detta för att minska risken för skador och sjukdomar hos människor.
            \item WEEE

                WEEE (Waste of Electrical and Electronic Equipment) är också en EU märkning. Ser man detta märke på en produkt betyder det att man inte får slänga den i osorterat avfall utan produkten måste lämnas till en särskild återvinningstation. Denna märkning finns på alla elektrisk och elektroniska utrustning som släpps på marknaden inom EU.
            \item Ecodesign

                "Ekodesigndirektivet sätter minimikrav på energiprestanda hos produkter och förbjuder de mest energi- och resurskrävande produkterna på EU-marknaden." Detta för att genom ett miljöperspektiv förbättra produktens prestanda under hela dess livscykel.
            \item REACH
                
            REACH (Registration, Evaluation, Authorisation and Restriction of Chemicals), är förordning från EU som används för att att hålla koll på produkter med kemiska ämnen. Dessa är det som REACH förkortningen står för. Det innehåller också krav på information du måste ge kunder om produkten innehåller dessa kemikalier. 
            \item EMC

            EMC (elektromagnetisk kompatibilitet) Direktivet handlar om att elektroniska produkter/utrustning inte ska störa varandra. Med EMC så menas att dessa olika produkter SKA fungera nära varandra utan att störa eller bli störda. Produkter som släpps på markanden i EU måste uppfylla de krav som finns för detta.
            \item LVD

            LVD (Low Voltage Directive) är ett direktiv som finns för att skydda människor genom ett högt skydd mot vissa spänningsintervall. Dessa spänningsintervall är 50, 1000 V för växelström och 75, 1500 V för likström.
            \item RED

            RED (Radio Equipment Directive) direktivet kan ses som ett ramverk för vilka radiokomunaktionsprodukter på marknaden. Detta ramverk ser till att väsentliga krav för säkerhet och hälsa följs, att de är EMC samt att de använder radiospektrumet effektivt. En annan viktig del med detta direktiv är kraven på skydd av privatlivet och integritet, personuppgifter och mot bedrägerier.
    
    \item SE.4 Redogör för de olika stegen i en produkts livscykel.

        
        
    \item SE.5 Att det finns komponenter tillgängliga under en produkts livstid är naturligtvis viktigt. Förklara vad som menas med följande för- kortningar och begrepp när det gäller en komponents livscykel:
    Samples, Qualification, NRND, LTB, Obsolete
    \item SE.6 Förklara vad som menas med IP och NDA när det gäller avtal inför tillverkning        


\end{itemize}

\end{itemize}

\section{Kapitel 7}

\begin{itemize}
    \item 7.1 Förklara hur en spektrumanalysator fungerar utgående från blockschema enligt Figur 7.2a i boken.

    En spektrumanalysator är en relativt billig mätningsverktyg som används för "snabb titt" när det kommer till testning och diagnoering. Den momentana spektrumdisplayen är extremt värdefull när det kommer säkerställa att bland annat frekvensen, då den enkelt kan "zooma" in på mindre delar av spektrumet. Den är också användbar för att kolla HF svar från kretsnät.  
    -------skriv mer ---- om genom block-------
    spännings funktion av frekvens
    byggd som radiomotagare,  behövs bra filter (bandpass)
    inputen blandas med signalen från en oscillator, för att titta på skillnaden. Efter mixen används ett bra filter som i en spektrumanalysator kan gå ner på få kilohertz, efter det finns en förstärkare, efter det en diod i form av en evelope detector, sen med ett filter kan vi hålla koll på medel/max värdet berode på vad vi vill mäta. Sen siganl proccessor för att visa på displat och vi styr med en "sweep generator". 

\item 7.2 Förklara vad som mäts med peak detector, average detector och quasi‐peak detector.

\begin{itemize}
    \item Peak
    Peak detektorn reagerar nästan omedelbart på signalens maxvärde och laddas ut snabbt. 

    \item Average
    Som namnet låter mäter en "average detector" genomnsittsvärde på signalen. För en kontinuerlig signal blir denna samma sak som peaken. 

    \item Quasi-peak
    Denna peak detektor fungerar med en viktande laddning och urladdningstid. Vilket korrigerar för det subjektivamännsilka svaret på interferensen av pulstyp. 
    
\end{itemize}

\item 7.3 Vad menas med antennfaktor?

Antennfaktor är en viktigt paramater när det kommer till EMC-mätningar. Detta anger hur effektivt en antenn omvandlar en elektromagnetisk våg till spänning vid antennens ingång. Detta via formlen \[AF=E-V\] Där AF = antennfaktorn och E = fältstyka samt V = volt för antennen.

\item 7.4 Hur kan man räkna om elektriska fältstyrkan som en antenn tar emot till elektrisk spänning på mätinstrumentets ingång?


\item 7.5 Hur definieras ett mätsystems känslighet (sensitivity)?

Känslighet definieras som det minsta mätbara signalvärdet som kan detekteras av system med tillräcklig noggrannhet. Ju högre känslighet desto mindre ingångssignaler kan detekteras.

\item 7.6 Vad betyder LISN och vad används den till?

LISN står för \textit{Line Impedance Stabilization Network}, används för att ge en definierad impedans vid radiofrekvsener (RF) vid en mätpunkt. Dess syfte är att koppla mätpunkten till testunstrumentering samtidigt som testkretsen isoleras från oönskade störningssignaler på strömförsörjningne. Denmest utbredda typen av LISN enligt CISPR 16-1-2 har en impedans som motsvarar 50 Ohm i parallelkoppling med en 50 mikroHenry + 5 Ohm över varje ledning i förhållande till jord.

\item 7.7 Rita figurer och beskriv hur man kan tillverka prober för att sniffa elektriskt respektive magnetiskt fält nära en störkälla.

Prober går att tillverka med hjälp utav en koaxialkabel själv allternativ att köpa ett kalibrerat set. Designen på proberna är avvägning mellan känslighetren och "spatial" noggrannhet. 
När man sniffar elektriska fält nära en strömkälla används en \textit{Short Rod Construction}, och för att sniffa magnetiska fält används \textit{Loop Construction}. 
---------------bild---------------------


\item 7.8 Beskriv hur emissionstester av ledningsbundna störningar respektive utstrålade störningar genomförs.

Test för ledningsbunda störningar är placeringen av EUT viktigt med avseende till jordplanet och LISN:et. Testerna görs i ett avskillt rum där väggar samt golv är jordat. LISN sitter mellan utrustning och vägguttag. 

När det kommer till tester av utstrålande störningstester, har man istället ett fast avstånd mellan mätantennen (speciell höjd) och EUT enheten.
\end{itemize}
\section{Kapitel 8}

\begin{itemize}
\item 8.1 Hur genomör man immunitetstester för utrålande störningar
respektive ledningsbunda störningar?

För ledningsbunda immunitetstester behöver mna ha utrustning som lägger på en störning över spänningsregulatorn. 

Immunitetsterster görs också i ett större avskärmat rum. EUT placeras i rummet
tillsammans med störningsutrustningen, så att störningarna inte når ut till
annan elektronik. Man sitter utanför med kontrollutrustning som skickar ut
signaler av olia frekvenser. EUT ska vara 
igång. Man mäter hur mycket effekt det är som ebhöbs vid en viss frekvens för att ett problem ska uppstå, som en blåskärm. 

\item 8.2 Vad betyder ESD?

	ESD står för \textit{Electrostatic discharge} vilket är en plöstslig ström av
	elektrisk laddning som kan uppkomma när två föremål med olika elektriskt 
	potential kommer i kontakt med varandra eller när en elektrisk laddaning 
	snabbt förändras. Det kan orsaka skador på elektroniska komponenenter, 
	och det är viktigt att ta lämliga försiktighetsåtgärder när man hanterar 
	elektronik för att förhindra ESD-skador. 

\item 8.3 Hur testar man immunitet mot elektrostatiska urladdningar?

	Man använder en ESD generator som har en \textit{discharge} elektrod som ska
	föra denna elektrostatsiska urladdning. Finns en chans att annan närliggande
	utrustning påverkas av dtta, så det råds att inte utföra ett sådant test på ett 
	"live" system. Gärs i ett skrämat rum, det är viktigt att ha ett jordat plan vid
	ett sådant test.
 
\end{itemize}

\section{kapitel 9-10}

läs boken översiktligt

\section{Kapitel 11}


\end{document}
